\documentclass{article}
\usepackage[utf8]{inputenc}
\usepackage[russian]{babel}
\usepackage{fontspec}
\setmainfont{CMU Serif}
\usepackage{amsmath}

\usepackage{titlesec}
\newcommand{\sectionbreak}{\clearpage}

\usepackage{hyperref}
\usepackage{minted}

\title{\textbf{ОТЧЁТ}\\
по лабораторной работе №4\\
``Шифр Вернама''\\
по специальности 01.03.02 Прикладная математика и информатика}

\usepackage{filecontents}
\begin{filecontents}{main.bib}
@article{vernam1926cipher,
  title={Cipher printing telegraph systems: For secret wire and radio telegraphic communications},
  author={Vernam, Gilbert S},
  journal={Journal of the AIEE},
  volume={45},
  number={2},
  pages={109--115},
  year={1926},
  publisher={IEEE}
}
@misc{keyreuse@se,
title = {Taking advantage of one-time pad key reuse?},
url = {https://crypto.stackexchange.com/questions/59/taking-advantage-of-one-time-pad-key-reuse}
}
\end{filecontents}

\usepackage{biblatex}
\addbibresource{main.bib}

\begin{document}
\begin{center}
МИНИСТЕРСТВО ОБРАЗОВАНИЯ И НАУКИ РОССИЙСКОЙ ФЕДЕРАЦИИ\\
ФЕДЕРАЛЬНОЕ ГОСУДАРСТВЕННОЕ БЮДЖЕТНОЕ ОБРАЗОВАТЕЛЬНОЕ УЧРЕЖДЕНИЕ\\
ВЫСШЕГО ПРОФЕССИОНАЛЬНОГО ОБРАЗОВАНИЯ\\
``ВОРОНЕЖСКИЙ ГОСУДАРСТВЕННЫЙ УНИВЕРСИТЕТ''\\
(ФГБОУ ВПО ``ВГУ'')\\
ФАКУЛЬТЕТ ПРИКЛАДНОЙ МАТЕМАТИКИ, ИНФОРМАТИКИ И МЕХАНИКИ\\
КАФЕДРА ВЫЧИСЛИТЕЛЬНОЙ МАТЕМАТИКИ И ПРИКЛАДНЫХ ИНФОРМАЦИОННЫХ ТЕХНОЛОГИЙ
\end{center}
\vfill
\begin{center}
\textbf{ОТЧЁТ}\\
по лабораторной работе №4\\
``Шифр Вернама''\\
по специальности 01.03.02 Прикладная математика и информатика
\end{center}
\vfill
\begin{flushright}
Выполнил:\\
студент 4-го курса 6-ой группы\\
Козлуков С.В.\\
Проверил:\\
доц. Воронков Б. Н.
\end{flushright}
\begin{center}
Воронеж\\2017
\end{center}
\clearpage
\tableofcontents
\clearpage

\section{Постановка задачи}
\begin{itemize}
\item Описать структуру шифра Вернама.
\item Зашифровать и расшифровать исходный текст, используя шифр Вернама.
\item Составить отчет о проделанной работе.
\end{itemize}

\section{Общие сведения}
Алгоритм шифрования Вернама был реализован
в компьютерной программе на языке Python3.

\section{Описание шифра Вернама}
Шифр Вернама --- потоковый one-time-pad шифр~\cite{vernam1926cipher}.
Пусть дан открытый текст \( {m: \{1,\ldots, N\} \to A: i\mapsto m_i} \)
длины \( N \) с элементами из алфавита
\( A=\{1,\ldots, M\} \).
Для шифрования требуется pre-shared ключ
\( k \) такой же длины \( N \).
Криптотекст определяется следующим образом:
\[
	c_i = m_i \oplus k_i.
\]
Для дешифрования требуется повторить операцию:
\[
	m_i = c_i \oplus k_i.
\]

\section{Пример работы программы}

\begin{minted}{python}
import random
random.seed(42)
\end{minted}

\begin{minted}{python}
msg = """It must be the law of diminishing returns...
I feel the spell about to be broken.
(Energizing himself somewhat.
He takes out a coin, spins it high, catches it,
turns it over on to the back of his other hand,
studies the coin ---
and tosses it to ROS.
His energy deflates and he sits.)
Well, it was an even chance...
if my calculations are correct"""
key = [random.randint(1, 255) for m in msg]
enc = 'ascii'
c = vernam_enc(msg, key, enc=enc)
\end{minted}

\begin{minted}{python}
bytes(c)
\end{minted}

\begin{minted}{python}
    b"\xedi'\xd32LN\x04\xdf~\x8e\xca\x8d\xe97\xf4\x0c~(w^\x1c\xe6\xf2j\xf9]\xd1\xd4\xdc\xe5\x02^S\xe5-\xa4\xaap\xad\xbc\x07\x9dCR\x01\x08^\x93\xa1;;l\n||\xaa)\xfe(\xa3,\xda\x14\xe5U\x8e\xcd\x15z\xae.\xb0\x81\xfd\x91\xb26\xf1\\\x9b\x18$\xefU\xa39\x9a|\xa1U\xb0}B \x1c\xce\xa5;F9{E\xc3(\xd1\x87\xc7\xc7g\xb2\xa9d\xec\x9bKK\x1c\x075\xde\x82\xd1\xc5\xafX\x907\xb7\xac\xa9#\x1b\xa0y\xa6?\x14exC\xca\x9a\xfb\x86\xd0}\x17\xcb\xe1\x12\x80\x83\xc0\x05\x05-Pl\xb5\xe4\xff6\xae\xe5N\x8f\xe2G2OA\xf1_w\xac-\xa9r\x08\xd5A\xae\x8f\x0f\xf8r\x08B\xf6\x1e\xa8)\x90\xfd\xfd\x9dw\xc7\xdcl\x8f\x8b\xeb\xaf!\xe9\xaf+i9\x14@\x10r\xd5\xcd\x89\xdcd\x9a\xee\xad@\xa2\xc71\xf2\xab,\xb6\xc0\xa2\xe8\\[\x13\xa1Y\xab\x9c\xbc\xcd\xfc\x83!\xc8\x1c-+\x17\xa64\x92\xdd\xb0\xa1*LhG\xc1\xf6\x96sz\xdd\t\xb4a\xdb\xa2\xe7\xa1\x01I\xccZ\x80\xe4_7\xa6\xc9\x96:\x92[\x82\xa7\xe2\xd2\xc5\x14\xc01\x14\xdf\xcd\xc9@\x14\x91\xe0\x1a?#Rp9e\xf2\xa0\x15\xb93ku\x96\xcfi\x1bq\x89e\xbe \x7f\xe5I!\xc3\x13D\xabC\xc8\x95\xc2\xf0\xfb\x08M\xac\x19\xbb"
\end{minted}

\begin{minted}{python}
print(vernam_dec(c, key, decode=True))
\end{minted}

\begin{verbatim}
    It must be the law of diminishing returns...
    I feel the spell about to be broken.
    (Energizing himself somewhat.
    He takes out a coin, spins it high, catches it,
    turns it over on to the back of his other hand,
    studies the coin ---
    and tosses it to ROS.
    His energy deflates and he sits.)
    Well, it was an even chance...
    if my calculations are correct
\end{verbatim}



\section{Выводы}
В ходе работы был изучен и реализован алгоритм Вернама.
Алгоритм обеспечивает идеальную криптостойкость,
однако неприменим в реальности из-за сложности
распределения ключей --- ключ должен быть
длины не меньшей длины открытого сообщения
и должен использоваться лишь единожды.
Также в реализации могут~\cite{vernam1926cipher}
возникнуть сложности
с согласованием кодировок текста.
Повторное использование ключа
(в том числе, "зацикливание" короткого ключа
для использования с длинным сообщением)
компроментирует~\cite{keyreuse@se} безопасность.

\section{Список использованных источников}
\printbibliography[heading=none]

\section{Код программы}
\begin{minted}{python}
import itertools

def vernam_enc(msg, key, enc='ascii', cycle_key=False, decode=False):
    if isinstance(msg, str):
        msg = msg.encode(enc)
    if isinstance(key, str):
        key = msg.encode(enc)
    N = len(msg)
    if len(key) < N and not cycle_key:
        raise ValueError('Vernam: `key` shall be as long as `msg`')
    key = itertools.cycle(key)
    msg = [m^k for m, k in zip(msg, key)]
    if decode:
        msg = bytes(msg)
        msg = msg.decode(enc)
    return msg

vernam_dec = vernam_enc
\end{minted}
\end{document}
